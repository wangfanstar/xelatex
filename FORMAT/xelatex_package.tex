
\usepackage{ctex}
%%\usepackage{ctexcap} % 需要将 ctexcap.sty 里与 ctex 宏包重复的部分注释,即不用加载相同包
\usepackage{relsize}                 % 调整公式字体大小:\mathsmaller, \mathlarger
%\usepackage{times}
%\usepackage{fontspec,xunicode,xltxtra} % XeLaTeX相关字体字库 xunicode会被ctex或fontspec自动载入
\usepackage{fontspec,xltxtra}
%%%%%%%%%%%%%%%%%%%%%%%%%%%%%%%%%%%%%%%%%%%%%%%%%%%

\usepackage{etex}  % 解决宏包 no room for 。。。的错误

%%%%%%%%%%% 版本修改记录宏包 %%%%%%%%%%%%%%%%%%%%%%
%\usepackage[nochapter]{vhistory}

\usepackage{morewrites} %解决 no room for \write 的错误
\usepackage{caption2}                               % 按标准, 去掉图表号后面的:
\usepackage{lipsum}   % To generate test text 产生测试文本

%%%%%%%%%%%%%% 颜色 %%%%%%%%%%%%%%%%%%%%%%%%%%%%%%%%%%%%%%
\usepackage[table,dvipsnames,svgnames]{xcolor}
\usepackage{xxcolor}

%%%%%%%%%%%%%% PGF绘图宏包 %%%%%%%%%%%%%%%%%%%%%%%%%%%%%%%%%%%%%%

%%%%%%%%%%%%%%%%%%%%%%% pgf 绘图 %%%%%%%%%%%%%%%%%%%%%%%%
\def\pgfsysdriver{pgfsys-dvipdfmx.def} %使用dvipdfmx的引擎,原XETEX生成图形有的有错误。
\def\xcolorversion{2.00}
\usepackage[version=latest]{pgf}

\usepackage{xkeyval,calc,tikz}
\usepackage{tikz-qtree}
%
\usetikzlibrary{
  arrows,
  calc,
  fit,
  patterns,
  plotmarks,
  shapes.geometric,
  shapes.misc,
  shapes.symbols,
  shapes.arrows,
  shapes.callouts,
  shapes.multipart,
  shapes.gates.logic.US,
  shapes.gates.logic.IEC,
  circuits.logic.US,
  circuits.logic.IEC,
  circuits.logic.CDH,
%  circuits.ee.IEC,
  datavisualization,
  datavisualization.formats.functions,
  er,
  automata,
  backgrounds,
  chains,
  topaths,
  trees,
  petri,
  mindmap,
  matrix,
  calendar,
  folding,
  fadings,
  shadings,
  spy,
  through,
  turtle,
  positioning,
  scopes,
  decorations.fractals,
  decorations.shapes,
  decorations.text,
  decorations.pathmorphing,
  decorations.pathreplacing,
  decorations.footprints,
  decorations.markings,
  shadows,
  lindenmayersystems,
  intersections,
  fixedpointarithmetic,
  fpu,
%  svg.path,
  external,
}

\tikzifexternalizing{%
}{%
\usepackage{pdfpages}
%\usepackage{vmargin}
}
%%%%%%%%%%%%%  电路宏包,更多电子元器件 %%%%%%%%%%%%%%%%%%%%%%%%%%
\usepackage[siunitx]{circuitikz}  %需加 etex package ,否则 supp-tex 有 no room for 。。。 的 error
\usepackage{tikz-timing}
\def\degr{${}^\circ$} %角度定义
\usetikztiminglibrary{advnodes}

%%%%%%%%%%%%%%%%%%%% 初始化
%
%\tikzset{external/system call={xelatex \tikzexternalcheckshellescape -halt-on-error-interaction=batchmode -jobname "\image" "\texsource"}}
%\tikzsetexternalprefix{figures/}% 设置图片保存路径
%\tikzexternalize %activate!


% Global styles:
\tikzset{
   every plot/.style={prefix=plots/pgf-},
   shape example/.style={
    color=black!30,
    draw=,
    fill=yellow!30,
    line width=.5cm,
    inner xsep=2.5cm,
    inner ysep=0.5cm}
}
\tikzset{
passprocess/.style={
rectangle,
draw=blue,
minimum width=50pt,
minimum height=20pt,
font=\ttfamily,
text centered
},
startstop/.style={
rectangle,%
rounded corners=5pt,%
minimum width=50pt,
minimum height=20pt,
text centered,
fill=orange,
font=\ttfamily,
draw=red
},
decision/.style={
diamond,%
shape aspect=2,%aspect value is the ratio of width and height for diamond
draw=green,%the color of line
fill=lime,%filled color
font=\ttfamily,%set font
text centered%surely you know what it means
},%here is a
line/.style = {
draw,
->,
%shorten>=2pt,
thick
}}




%\usepackage[active,tightpage]{preview}
%\setlength\PreviewBorder{5pt}%


%%%%<
%\PreviewEnvironment{tikzpicture}
%%%%>

\tikzset{
  paint/.style={draw=#1!50!black, fill=#1!50},
  information text/.style={rounded corners,fill=red!10,inner sep=0ex},
  my star/.style={decorate,decoration={shape backgrounds,shape=star},
                  star points=#1}
}


%%%%%%%%%%%%%  坐标图绘制宏包%%%%%%%%%%%%%%%%%%%%%%%%%%
\usepackage{pifont}
\usepackage{pgfplots}
\pgfplotsset{width=7cm,compat=1.4}

\usepgfplotslibrary{%
	ternary,
	smithchart,
	patchplots,
	polar,
	colormaps,
}

%%%%%%%%%%%%% 动画设置 %%%%%%%%%%%%%%%%%%%%%%%%%%

\tikzset{overlap/.style={fill=yellow!30},
    block wave/.style={thick},
    function f/.style={block wave, red!50},
    function g/.style={block wave, green!50},
    convolution/.style={block wave, blue!50},
    function g position/.style={function g, dashed, semithick},
    major tick/.style={semithick},
    axis label/.style={anchor=west},
    x tick label/.style={anchor=north, minimum width=7mm},
    y tick label/.style={anchor=east},
}
\pgfkeys{/pgf/number format/.cd,fixed,precision=1}

\pgfdeclarelayer{background}
\pgfdeclarelayer{foreground}
\pgfsetlayers{background,main,foreground}




%%%%%%%%%%%%% yellownote 边注设计 %%%%%%%%%%%%%%%%%%%%%%%%%%

\newlength{\yellownotewidth}
\setlength{\yellownotewidth}{2cm}
\newlength{\yellownoteheight}
\setlength{\yellownoteheight}{2cm}
\newcommand{\yellownote}[1]{
\marginpar{
    \vspace{-0.5\yellownoteheight}
        \begin{center}
        \begin{tikzpicture}
            \draw[white,fill=gray!25,opacity=0.75,shift={(-0.125,-0.125)}]
                (0,0) rectangle (\yellownotewidth,\yellownoteheight);
            \draw[fill=yellow!35] (0,0) rectangle (\yellownotewidth,\yellownoteheight);
            \draw[opacity=0.45,fill=gray!50] (0.7\yellownotewidth,0) --
                (0.9\yellownotewidth,0.45) -- (\yellownotewidth,0.4) -- cycle;
            \node[blue,below] at (0.5\yellownotewidth,\yellownoteheight) {
                \begin{minipage}{\yellownotewidth-1em}
                    \scriptsize\sf#1
                \end{minipage}
            };
        \end{tikzpicture}
        \end{center}
        \vspace{0.5\yellownoteheight}
    }
}

%   -   -   -   -   -   -   -   -   -   -   -   -
% Resizeable - Yellow note...
%   -   -   -   -   -   -   -   -   -   -   -   -
\newcommand{\resizeyellownote}[3]{
\setlength{\yellownotewidth}{#1cm}
\setlength{\yellownoteheight}{#2cm}
\marginpar{
    \vspace{-0.5\yellownoteheight}
        \begin{center}
        \begin{tikzpicture}
            \draw[white,fill=gray!25,opacity=0.75,shift={(-0.125,-0.125)}]
                (0,0) rectangle (\yellownotewidth,\yellownoteheight);
            \draw[fill=yellow!35] (0,0) rectangle (\yellownotewidth,\yellownoteheight);
            \draw[opacity=0.45,fill=gray!50] (0.7\yellownotewidth,0) --
                (0.9\yellownotewidth,0.45) -- (\yellownotewidth,0.4) -- cycle;
            \node[blue,below] at (0.5\yellownotewidth,\yellownoteheight) {
                \begin{minipage}{\yellownotewidth-1em}
                    \scriptsize\sf#3
                \end{minipage}
            };
        \end{tikzpicture}
        \end{center}
        \vspace{0.5\yellownoteheight}
    }
}





%%%%%%%%%%%% 合并PDF文档 与tikz extern 冲突%%%%%%%%%%%%%%%%%%%%%%
%\usepackage{pdfpages}

%%%%%%%%%%%% 图表标题格式包 %%%%%%%%%%%%%%%%%%%%%%
\usepackage[Euler]{upgreek}
\usepackage{amsmath,amsfonts,amssymb} %
\usepackage{latexsym,bm}        %公式符号
\usepackage[misc,electronic,clock]{ifsym} %电气符号
%\usepackage{dingbat} % \checkmark 与 ams 冲突
\usepackage[Omega,upmu]{gensymb}
\usepackage{wasysym}
%\usepackage{marvosym}  % \letter 与 ifsym 冲突

%%%%%%%%%%%%%%%%    插图  %%%%%%%%%%%%%%%%%%%%%%%%%%%%%%%%%%%%
\usepackage{graphicx}       %插图宏包
\usepackage{wallpaper}    %绘图文绕排宏包,页面背景宏包,
\usepackage{picinpar} %

%%%%%%%%%%%%%% 彩色表格,表格线条 %%%%%%%%%%%%%%%%%%%%%%%%%%%%%%%%%%%%
\usepackage{tabu}
\usepackage{booktabs,colortbl,diagbox,longtable,multirow,tabularx,dcolumn}                         %表格粗线,斜线,彩色表格,长表格
%%%%%%%%%%%%% 页版面,边距设置 %%%%%%%%%%%%%%%%%%%%%%%%%%%%%%%%%%
\usepackage[top=2.54cm,bottom=2.54cm,left=2.15cm,right=2.5cm,includehead,includefoot]{geometry}
%上下2.54,左右2
%%%%%%%%%%% 中文书签中文复制 %%%%%%%%%%%%%%%%%%%%%%%%%%%%%%%%%%
\usepackage[colorlinks=no,
            citecolor=blue,
            linkcolor=blue,
            anchorcolor=green,
            urlcolor=blue,
%% 与attachfile2冲突           pdfauthor={wangfan},%作者
%%            pdfkeywords={latex},%关键词
%%            pdfsubject={latex},%主题
%%            pdftitle={handbook of latex},%标题
            CJKbookmarks=true,
            pdfborder={0 0 0},
            bookmarksnumbered=true,
            bookmarksopen=false,
            xetex,
            ]{hyperref}
%\usepackage{ccmap}               % 使生成的PDF文件支持复制等,对pdflatex
%
%
%%%%%%%%%%%%%%%%%%%%%%%%%%%%%%%%%%%%%%%%%%%%%%%%%%%%%%%%%%%%%%%%%%%
\usepackage{titletoc}           %目录格式包


%%%%%%%%%%%%%%%%%%%%%%%%%%%%%%%%%%%%%%%%%%%%%%%%%%%%%%%%%%%%%%%%%%%标题中文化
\usepackage[bf,small,raggedright,indentafter,pagestyles]{titlesec}
        %其中bf设置章节标题的字体为黑体,这也是默认值,可以略去。
        %此外,还可以设 为rm(罗马体), sf(无衬线体), tt(打字机体), md(中等黑度),
        %up(直立体), it(意大利斜体), sl(机械斜体), sc(小体大写字母)。
        %small设置标题字体的尺寸,还可设为big(默认), medium, tiny。
        %center使标题居中,还可以设为raggedleft(居左,默认), raggedright(居右)。
        %indentafter相当于宏包indentfirst的作用,使标题下面的第一个段落正常缩进。
        %pagestyles是申明后面要自定义页面样式。

%%%%%%%%%%%%%%%%%%%%%%%%%%%%\usepackage{tocloft}

\usepackage{fancyhdr}       %自定义页眉页脚

%%%%%%%%%%%% 抄录环境 %%%%%%%%%%%%%
\usepackage{fancyvrb,sverb}
%
\input{FORMAT/detail/asysyntax.tex} %listings语法高亮设置


%%%%%%%%%%%%%% 页码宏包 (与动画宏包冲突)%%%%%%%%%%%%%%%%%%%%%%
\usepackage{lastpage}
%\usepackage{fancybox} %与framed宏包冲突

%%%%%%%%%%%% 盒子环境 %%%%%%%%%%%%%
\usepackage{framed}
\usepackage{mdframed}
\usepackage[listings,theorems,skins]{tcolorbox}
\tcbset{noparskip}
%%%%%%%%%%%%% ASY绘图宏包 %%%%%%%%%%%%%%%%%%%%%%
\usepackage{asymptote}

%%%%%%%%%%%%% SHAPE宏包 %%%%%%%%%%%%%%%%%%%%%%
\usepackage{shapepar}

%%%%%%%%%%%%% 图片放置宏包 不放在文字前面 %%%%%%%%%%%%%%%%%%%%%%
\usepackage{flafter,float}

%%%%%%%%%%%%% 下划线宏包 %%%%%%%%%%%%%%%%%%%%%%
 \usepackage[normalem]{ulem}%`加入宏包`
 \usepackage{CJKfntef} %汉字下划线宏包



%%%%%%%%%%%%% 动画宏包 %%%%%%%%%%%%%%%%%%%%%%
\usepackage{animate} % 与 tikz 部分宏包冲突

%%%%%%%%%%%% 行号宏包 %%%%%%%%%%%%%%%%%%%%%%
\usepackage[left]{lineno} %与 tikz 宏包冲突

%%%%%%%%%%%%% 视频宏包 %%%%%%%%%%%%%%%%%%%%%%
\usepackage{movie15} % 与 tikz 部分宏包冲突

%
%%%%%%%%%%%%% 时间宏包 %%%%%%%%%%%%%%%%%%%%%%
%%%\usepackage{tdclock}  与 pdfcomment 冲突


%%%%%%%%%%%%% 短列表宏包 %%%%%%%%%%%%%%%%%%%%%%
%\usepackage{shortlst}
%
%%%%%%%%%%%%% 列表编号宏包 %%%%%%%%%%%%%%%%%%%%%%
\usepackage{enumerate}



%%%%%%%%%%%% 脚注尾注宏包 %%%%%%%%%%%%%%%%%%%%%%
\usepackage{threeparttable,endnotes}

%%%%%%%%%%%% 索引表 %%%%%%%%%%%%%%%
\usepackage{makeidx}\makeindex

%%%%%%%%%%% 索引宏包 %%%%%%%%%%%%%%%%%%%%%%
%\usepackage{xesearch}
%\usepackage{xeindex}\makeindex


%%%%%%%%%%%% 引用包 %%%%%%%%%%%%%%%
\usepackage{cite}  %实现[1-4]方式引用多个参考文献

%%%%%%%%%%%% 双栏排版宏包 %%%%%%%%%%%%%%%
\usepackage{flushend,cuted}
\usepackage{multicol} %多栏排版
%
%%%%%%%%%%%%%% 生成HTML宏包 %%%%%%%%%%%%%%%%%%%%%%
%%\usepackage{html,epsf}


%%%%%%%%%%% 附件宏包 %%%%%%%%%%%%%%%%%%%%%%
\usepackage{attachfile2}

%%%%%%%%%%% 目录结构图宏包 %%%%%%%%%%%%%%%%%%%%%%
\usepackage{dirtree}

%%%%%%%%%%% 柱状图宏包 %%%%%%%%%%%%%%%%%%%%%%
\usepackage{bardiag}

\usepackage{drawstack}
%%%%%%%%%%% 书签宏包 %%%%%%%%%%%%%%%%%%%%%%
\usepackage[open,openlevel=0,atend]{bookmark}



%%%%%%%%%%% pdf 注释宏包 %%%%%%%%%%%%%%%%%%%%%%
\usepackage[subject={tex},author={wangfan},dvipdfmx,version=1]{pdfcomment}
