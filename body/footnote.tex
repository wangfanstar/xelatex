\section{脚注边注设计}
\index{命令!\verb$\footnote$}

\subsection{脚注编号 footnote}
\subsubsection{使用方法}
\begin{lstlisting}[language={[LaTeX]TeX}]
\footnote[`序号`]{`脚注内容`}
\end{lstlisting}

\subsubsection{调整脚注序号}
不带序号的脚注前加以下命令:
\begin{lstlisting}[language={[LaTeX]TeX}]
\renewcommand{\thefootnote}{}%`清除脚注序号`
\end{lstlisting}
序号改为字母形式:
\begin{lstlisting}[language={[LaTeX]TeX}]
\renewcommand{\thefootnote}{\alph{footnote}}
\end{lstlisting}
脚注序号是以每章第一个脚注为 1 开始排序,要以全文为单位排序可用
remmreset 宏包,并在后加入以下命令:
\begin{lstlisting}[language={[LaTeX]TeX}]
\makeatother
\@removefromreset{footnote}{chapter}
\makeatother
\end{lstlisting}
\subsubsection{调整脚注字体}
默认为中文宋体,英文罗马体,尺寸为 footnotesize。
可用下述命令修改:
\begin{lstlisting}[language={[LaTeX]TeX}]
\renewcommand{\footnotesize}{\small\sf\fangsong}
\end{lstlisting}

\subsubsection{小页环境 minipage 中的脚注}
\index{命令!\verb$\footnotemark$}
脚注在小页的底部,而不是页面底部。小页环境中脚注序号为小写英文字母,计数器为 mpfootnote。

如果要求小页脚注与普通脚注统一,可用 \verb|\footnotemark| 和  \verb|\footnotetext{}| 来插入脚注,
 \verb|\footnotemark| 插在要标注的地方, \verb|\footnotetext{}|放在小页环境后。括号内为脚注内容。

\subsubsection{表格中的脚注}
tabular 和 array 不支持脚注命令\verb|\footnote| ,longtable 和 tabularx 表格环境中可使用,脚注在页底,
如要在表格下实现脚注,可将 longtable 或 tabularx 嵌入 minipage 中。

但实现的表格脚注长度可能超过表格宽度,因此可使用宏包 threeparttable 来实现表格环境的脚注,可实现三种表格环境
tabluar、tabluarx、tabluar$*$ 三种环境的表格脚注。
使用要点:threeparttable 环境是不浮动的,如果要浮动,可再置于 table 或 table$*$ 环境中。
\index{命令!\verb$\begin{threeparttable}$}
\index{宏包!threeparttable}
\begin{lstlisting}[language={[LaTeX]TeX}]
\usepackage[]{threeparttable}
%`可选参数有:`
%para `所有脚注作为一段,默认每个脚注一段`
%flushleft  `脚注左侧对齐,默认缩进 1.5em`
%online  `标记与标注同尺寸,默认为上标形式`

\tnote{`标记序号`}
%`在表格后用以下代码`
\begin{center}
\begin{threeparttable}
\caption{`表格标注`}
\begin{tabular}{ll}
\toprule
`我是表格`\tnote{a} & `是什么表格` \\
`表` & `测试数据,加长表格` \tnote{b}\\
\bottomrule
\end{tabular}
\begin{tablenotes} \footnotesize
  \item[a] `我是表格标注1`
  \item[b] `我是表格标注2`
\end{tablenotes}
\end{threeparttable}
\end{center}
\end{lstlisting}

\begin{center}
\begin{threeparttable}
\caption{表格标注}
\begin{tabular}{ll}
\toprule
我是表格\tnote{a}&是什么表格\\
表&测试数据,加长表格\tnote{b}\\
\bottomrule
\end{tabular}
\begin{tablenotes} \footnotesize
  \item[a] 我是表格标注1
  \item[b] 我是表格标注2
\end{tablenotes}
\end{threeparttable}
\end{center}

\subsection{尾注 endnotes}
\index{命令!\verb$\endnote$}
\index{命令!\verb$\theendnotes$}
\index{宏包!endnotes}

尾注宏包 endnotes\endnote{尾注测试1} 将标记内容统一放在一个地方\endnote{尾注测试2} ,一章后或书最后:

注意:theendnotes 如放在一章的中间后会引起 PDF 书签的混乱,之后页眉章号会全变成 notesname。
\begin{lstlisting}[language={[LaTeX]TeX}]
\usepackage{endnotes}
\endnote{`尾注内容`}
\theendnotes
`常用方法`
\renewcommand{\notesname}{`本章注释`}
\theendnotes
\setcounter{endnote}{0}
\end{lstlisting}



\subsection{边注 marginpar}

\index{命令!\verb$\marginpar$}
\begin{shaded}
  \begin{Verbatim}
    \marginpar[左边注内容]{右边注内容}
  \end{Verbatim}
\end{shaded}
\begin{enumerate}
  \item 单页排版时写入右边注
  \item 双页排版写入左页左边,右页右边
  \item 边注不能自动分页
\end{enumerate}
