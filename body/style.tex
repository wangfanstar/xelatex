\section{版式设计}
\subsection{常用文档类别 book article}
可选 4 种文档类别。
\begin{itemize}
  \item book
  \item report
  \item article
  \item letter
\end{itemize}
可选参数有字体尺寸,纸张大小,纸张方向,有无标题页
\begin{cmd}[label= 基本字体尺寸]
10pt,11pt,12pt
\end{cmd}
默认 10pt。
\begin{cmd}[label= 纸张大小,方向]
a4paper (297 X 210 mm)   executivepaper(10.5 X 7.25 in)
a5paper (210 X 148 mm)   legalpaper(14 X 8.5 in)
b5paper (250 X 176 mm)   letterpaper(11 X 8.5 in)
默认纵向,可选 landscape 参数为横向
\end{cmd}
对于 book 和 report ,默认标题页单独一页,加入 notitlepage 参数可使标题与开头文本在同一页。
对于 article 类,默认标题与正文同一页,加入 titlepage 参数可使标题单独一页。
\begin{cmd}[label= 标题页选项]
notitlepage     titlepage
\end{cmd}

\subsection{标题格式设计 caption2宏包}
\index{宏包!caption2}

\begin{description}
\item [normal] 标题文本两边对齐,其中最后一行为左对齐。
\item [center] 标题文本居中。
\item [flushleft] 标题文本左对齐。
\item [flushright] 标题文本右对齐。
\item [centerlast] 标题文本两边对齐,其中最后一行居中。
\item [indent] 与~\textbf{normal}~式样相似,只是标题文本从第二行开始,
               每行行首缩进由命令~\verb|\{captionindent}|~给出的长度。因为
               ~\verb|{captionindent}|~的缺省值为零,通常用像~
              \verb|\setlength{\bs captionindent}{1cm}|~这样的命令
              来设置缩进值。
\item [hang] 与~\textbf{normal}~式样相似,只是标题文本从第二行开始,
             每行行首缩进与标题标记宽度相等的长度。
\end{description}

通常这些标题式样在调入宏包时给出,如:
\begin{Verbatim}[xleftmargin=2cm]
\usepackage[centerlast]{caption}
\end{Verbatim}

将使整个文档中的标题都为~\textbf{centerlast}~式样。


\begin{table}
\newcommand{\tbltt}[1]{\textcolor{cyan}{\texttt{#1}}}
\renewcommand{\arraystretch}{1.2}
\centering
\caption{caption2~{\hei 选项}}\label{tab:caption2opt}
\begin{tabular}{|>{\kai\color{blue}}m{2cm}|m{3cm}|>{\kai}p{\textwidth - 6.5cm}|}
\hline
标题式样 & \tbltt{normal, center, flushleft, flushright, centerlast,
  hang, indent} & 选择标题的式样。 \\
\hline
标题字号 & \tbltt{scriptsize, footnotesize, small, normalsize, large, Large}
  & 选择标题的标记和文本的字体大小。\\
\hline
标记字形 & \tbltt{up, it, sl, sc} & 选择标题的标记的字形,不会影响到
  标题的文本。 \\
\hline
字体序列 & \tbltt{mb, bf} & 选择标题的标记的字体序列,即字体的宽度或
  权重。不会影响到标题的文本。\\
\hline
标记字族 & \tbltt{sl, sf, tt} & 选择标题的标记的字族,可为~Roman,
   San Serif~或~Typewriter~字体。不会影响到标题的文本。 \\
\hline
单行标题 & \tbltt{oneline, nooneline} & 控制是否采用单行标题格式     \\
\hline
\end{tabular}
\end{table}

\index{命令!\verb$\titlefromat$}
\scriptsize
\begin{lstlisting}
%%%% 定义段落章节的标题和目录项的格式  %%%%%
\titleformat{\chapter}[hang]{\centering\LARGE\bfseries}{\chaptername}{1em}{}
\setcounter{secnumdepth}{4} \setcounter{tocdepth}{4}
\titlespacing{\chapter}{0pt}{2.4ex}{2.4em}
\titleformat{\section}[hang]{\color{blue}\large\bfseries}{\thesection}{1em}{}
\titleformat{\subsection}[hang]{\color{grass}\large\bfseries}{\thesubsection}{1em}{}
\titleformat{\subsubsection}[hang]{\color{lightblue}\large\bfseries}{\thesubsubsection}{1em}{}
\end{lstlisting}
\normalsize
\index{命令!\verb$\titlecontents$}
本文档的目录格式设计:
\scriptsize
\begin{lstlisting}
\titlecontents{chapter}
[0em]
{\color{blue}\heiti\large\heiti\addvspace{1.5ex}}
%[0em] 为目录距左边的距离 \addvspace 为章与章之间的行距
{\xiaosi\thecontentslabel{~~~}}
{}          %{2em} 为章号距章标题的距离{}为章标题前的内容
{\titlerule*[0.5pc]{.}\contentspage}[\addvspace{0.5ex}] %0.5ex 为章到下节的距离

\titlecontents{section}
[2em] {\color{blue}\normalsize\addvspace{0.5ex}}
{\thecontentslabel\hspace*{1em}} {\hspace*{-2.3em}}
{\titlerule*[0.4pc]{.}\contentspage}

\titlecontents{subsection}
[4em] {\color{grass}\normalsize\addvspace{0.5ex}}
{\thecontentslabel\hspace*{1em}} {\hspace*{-2.3em}}
{\titlerule*[0.4pc]{.}\contentspage}

\titlecontents{subsubsection}
[6em] {\color{grass}\normalsize\addvspace{0.5ex}}
{\thecontentslabel\hspace*{1em}} {\hspace*{-2.3em}}
{\titlerule*[0.4pc]{.}\contentspage}
\end{lstlisting}
\normalsize

\subsection{多目录格式设计 minitoc宏包}
\index{命令!\verb$\minitoc$}
\index{宏包!minitoc}
可以用来给每章,节前加一个小目录。

\subsection{页边距设计 geometry宏包}
\index{宏包!geometry}
\begin{cmd}
  \usepackage[top=2.54cm,bottom=2.54cm,left=2cm,
  right=2cm,includehead,includefoot]{geometry}
%上下2.54,左右2
\end{cmd}

\subsection{空页面,页眉页脚}
\index{命令!\verb$\pagestyle{}$}
pagestyle的版式有 4 种\\
\begin{enumerate}
  \item plain : 默认 页眉空,页码居中在页脚
  \item empty : 空白页
  \item headings : 页眉由页码和页眉标题,每一章的第一页不显示页眉
  \item myheadings : 同 headings ,但页眉标题不是自动提取,由
  markright 和 markboth 决定。
\end{enumerate}

\begin{lstlisting}
\pagestyle{empty} \thispagestyle{empty}
\end{lstlisting}


myheadings 中用 $\backslash$markright\{右页页眉\}
$\backslash$markboth\{左页页眉\}\{右页页眉\}来指定内容,这些内容都保存在
leftmark 和 rightmark 里

\subsection{自定义页眉页脚}
%%%%%%%%%%%%%%%%%% 索引 %%%%%%%%%%%%%%%%%%%%
\index{宏包!fancyhdr}
\index{命令!\verb$\fancyhf$}
\index{命令!\verb$\fancyfoot$}
\index{命令!\verb$\fancyhead$}
用到 fancyhdr 宏包,book 文档 leftmark 代表章标题,rightmark
代表节标题。
\index{命令!\verb$\thispagestyle{}$}
\begin{lstlisting}[language={[LaTeX]TeX}]

\usepackage{fancyhdr}
\pagestyle{fancy} \thispagestyle{fancy}

\fancyhf{} //`清空页眉页脚` \lhead{}//`左页眉内容` \rhead{} \lfoot{}
\cfoot{} \rfoot{}

\renewcommand{\headrulewidth}{0.6pt}//`上下线的粗细`
\renewcommand{\footrulewidth}{0.6pt}
\pagenumbering{Roman}//`页码格式`

\fancyhead[L,O]{}//[]`内为L,C,R和O,E,左中右和奇偶页的组合`
\fancyfoot[]{}

\end{lstlisting}

\subsection{格式切换}

\index{命令!\verb$\backmatter$}
\index{命令!\verb$\frontmatter$}
\index{命令!\verb$\mainmatter$}

通常是书籍类的正文,前,后三种不同的格式,
主要是改变页码和章的计数。
\begin{enumerate}
  \item $\backslash$frontmatter : 把页码换成罗马数字格式,不对章进行自动编号,放在前言和目录前
  \item $\backslash$mainmatter : 把页码换成阿拉拍数字,对章进行自动编号,放在正文前
  \item $\backslash$backmatter : 不改变页码格式,不对章进行自动编号,放在参考文献前
\end{enumerate}


\subsection{文本格式shapepar}

\index{命令!\verb$\heartpar$}
\index{命令!\verb$\diamondpar$}
\index{命令!\verb$\squarepar$}
\index{命令!\verb$\nutpar$}

可以将文本排成心形,钻石形,坚果形。
\begin{lstlisting}[language={[LaTeX]TeX}]

\usepackage{shapepar}
\heartpar{`内容`} \diamondpar{`内容`} \squarepar{`内容`}
\nutpar{`内容`}
\end{lstlisting}
\color{red}
\heartpar{
我我我我我我我我我我我我我我我我我我我我我我我我我我我
我我我我我我我我我我我我我我我我我我我我我我我我我我我
我我我我我我我我我我我我我我我我我我我我我我我我我我我
我我我我我我我我我}
~\\
\diamondpar{我我我我我我我我我
我我我我我我我我我我我我我我我我我我我我我我我我我我我
我我我我我我我我我我我我我我我我我我我我我我我我我我我
我我我我我我我我我我我我我我我我我我我我我我我我我我我}

~\\

\nutpar{我我我我我我我我我
我我我我我我我我我我我我我我我我我我我我我我我我我我我
我我我我我我我我我我我我我我我我我我我我我我我我我我我
我我我我我我我我我我我我我我我我我我我我我我我我我我我}
\normalcolor


\clearpage

