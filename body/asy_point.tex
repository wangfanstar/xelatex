\subsection{点的绘制}
%
\subsubsection{空心点,实心点}
dot(参数)表示画实心点;
dot(参数,UnFill)表示画空心点;
参数一般为 pair 二元组的数据类型,表示平面坐标。\\

\lstinputlisting{body/asycode/dots.asy}

\subsubsection{网络格点}
1.自己绘制

\begin{lstlisting}
//`背景网格`
for (int i = 0; i <= 8; ++i) {
  real x = i * cm;
  //`横线`
  draw((0,x)--(8cm,x));
  //`竖线`
  draw((x,0)--(x,8cm));
}
\end{lstlisting}

2.调用 math 模块的 grid 函数\\
\begin{lstlisting}
import math;
add(grid(10,10,gray));
\end{lstlisting}
grid函数使用方法:\\
\begin{lstlisting}
 picture grid(int Nx, int Ny, pen p=currentpen)
\end{lstlisting}


以上为绘制一个 Nx X Ny ,间距为 1 的图形,为 pic 格式。\\
\begin{lstlisting}
add(grid(10,10,gray));
\end{lstlisting}

要使用 grid 函数画的图形,要使用 add(图) 命令,把这个图形加在当前的图上:



\subsubsection{比例分点,中点}
可用这种形式:\textcolor[rgb]{0.00,0.50,0.00}{interp(A,B,t)}
来表示比例分点,其中 t 为比例因子为 real 类型;A,B 为点坐档,pair类型。\\
用 midpoint 函数它们的中点。调用格式是:\textcolor[rgb]{1.00,0.00,0.00}{midpoint(path)},代码如下所示:\\
\begin{lstlisting}
pair X=interp(A,B,t);
pair D=midpoint(A- -B)
\end{lstlisting}


\subsubsection{交点}
调用函数 extension:\\
\begin{lstlisting}
 pair extension(pair P, pair Q, pair p, pair q);
\end{lstlisting}
%
返回线段P\,-\,-\,Q与p\,-\,-\,q延长线的交点,否则,如果两直线平行,返回(infinity,infinity)。

\clearpage
