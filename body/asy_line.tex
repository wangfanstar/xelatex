%\subsection{线的绘制}
%
%\subsubsection{实线,虚线}
%代码如下:\\
%
%  \lstinputlisting{body/asycode/lines.asy}
%
%  \begin{description}
%    \item[solid] 可以省略, 表示画实线;
%    \item[dashed ]  画虚线
%    \item[dashdotted] 画点划线
%    \item[dotted ]  画实心点线
%  \end{description}

%\subsubsection{箭头线}
%代码如下:\\
%
%  \lstinputlisting{body/asycode/arrows.asy}
%
%
% \begin{description}
%    \item[Arrow,EndArrow] 效果一样, 都是在路径的末端添加箭头;
%    \item[BeginArrow] 在路径的开头加箭头
%    \item[Arrows] 在路径的头尾都加上箭头
%    \item[MidArrow]  在路径的中间添加箭头
%  \end{description}
%
%\subsubsection{曲线,函数曲线}
%调用 graph 函数,返回类型为 path(guide) 的路径,调用代码格式为:\\
%\begin{center}
%\fcolorbox{white}{lightgreen}{\parbox{12cm}{
%real f(real x)\{return y=x\^{}2;\};\\
%//real x 声明函数自变量 x 是实数型,\\
%//f 前面的 real 声明函数 f 也是一个实数型.\\
%guide graph(real f(real), real a, real b,interpolate join=operator - -);\\
%//描述函数曲线
%  }}
%\end{center}
%
%
%其中:\begin{description}
%        \item[曲线的函数] real f(real x){return y=x}
%        \item[变量范围] a,b
%        \item[画笔线型] 折线:operator - - ;曲线:operator ..
%      \end{description}
%
%  \lstinputlisting{body/asycode/graph_line.asy}\\
%
%\clearpage
