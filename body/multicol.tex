\twocolumn

\section{多栏排版}
\index{命令!\verb$\onecolumn$}
\index{命令!\verb$\twocolumn$}
\subsection{onecolumn,twocolumn 选项}

在  documentclass 的可选项中有 onecolumn 和 twocolumn 选项,twocolumn 的参数调节命令有
\begin{cmd}[label= twocolumn 参数命令]
1.栏间距
\setlength{\columnsep}{宽度}
2.分隔线(默认宽度为0,不显示)
\setlength{columnseprule}{宽度}
3.栏宽(类似于\textwidth,可使用)
\columnwidth
\end{cmd}

\begin{enumerate}
  \item 以上命令放在导言区,可作用于全文,放在正文中,只作用于局部
  \item 只在部分页面中使用两栏格式,不用twocolumn选项,而应在正文中用\\ \verb|\twocolumn| 新起一页,遇到 \verb|\onecolumn| 恢复单栏。
  \begin{cmd}
    \twocolumn[通栏文本]
    \onecolumn
  \end{cmd}
  \item 用 flushend 和 cuted 宏包可实现在页面左右栏平衡排版和一个页面中既有单栏又有双栏。
\end{enumerate}

\subsection{flushend,cuted 平衡双栏,调整双栏宏包}
\index{宏包!flushend}
\index{宏包!cuted}
\index{环境!strip}
\index{命令!\verb$\raggedend$}
\index{命令!\verb$\flushend$}
默认是左右栏等高处理,可用以下命令调整
\begin{cmd}
1.宏包
\usepackage{flushend,cuted}
2.先排满左栏,再排右栏(默认平衡排版)
\raggedend
3.恢复平衡双栏命令
\flushend
4.插入单栏内容
\begin{strip}
  单栏内容
\end{strip}
\end{cmd}
\onecolumn
\subsection{multicol 多栏排版宏包}
\index{宏包!multicol}
\index{环境!multicols}
多栏输出,或分栏时不另起一页。
\begin{lstlisting}
\begin{multicols}{`分栏数字`}
`文本`
\end{multicols}
\end{lstlisting}

\begin{cmd}
按章节内容分栏
\begin{multicols}{3}[\section{The User Interface}]
分栏前加入单独的长文本 This index contains...
\begin{multicols}{3}[\section{Index}This index contains...][6cm]

\columnsep 列间距
分割线参数
\columnseprule :线宽 .4pt
\columnseprulecolor : 颜色,默认为 \normalcolor
\raggedcolumns : 底部对齐
\flushcolumns :默认对齐方式
\columnbreak :切换栏数
\premulticols :
\postmulticols :
\unbalanced :
\end{cmd}

