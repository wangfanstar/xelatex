\section{下划线}
\subsection{ulem 宏包}
\index{命令!\verb$\uline$}
\index{命令!\verb$\uuline$}
\index{命令!\verb$\uwave$}
\index{命令!\verb$\sout$}
\index{命令!\verb$\xout$}
\index{宏包!ulem}

使用方式,因为 ulem 宏包重新定义了加粗 $\backslash$emph
命令,为避免加粗定义被取消,用[normalem]选项来加入宏包。
\begin{lstlisting}[language={[LaTeX]TeX}]
 \usepackage[normalem]{ulem}%`加入宏包`
 \uline{text} %`下划单线`
 \uuline{text}%`双横线`
 \uwave{text}%`波浪线`
 \sout{text}%`中间删除线`
 \xout{text}%`斜线阴影状`
\end{lstlisting}

\fbox{ \uline{text}  \uuline{text} \uwave{text} \sout{text}
\xout{text}}

%\subsection{CJKfntef 宏包}
%\index{命令!\verb$\CJKunderline$}
%\index{命令!\verb$\CJKunderdot$}
%\index{命令!\verb$\CJKunderwave$}
%\index{命令!\verb$\CJKunderdbline$}
%\index{命令!\verb$\CJKsout$}
%\index{命令!\verb$\CJKxout$}
%\index{宏包!CJKntef}
%
%下划线可换行,可与 color
%宏包嵌套,CJKunderline的一个用途是利用全角空格产生可自动分行的空下划线,可用于填空题。:
%\begin{enumerate}
%    \item 分散对齐
%    \item 加点
%    \item 下划线
%\end{enumerate}
%
%\begin{lstlisting}[language={[LaTeX]TeX}]
% \usepackage{CJKfntef}%`加入宏包`
% \CJKunderline{`文 本`} %`下划单线`
% \CJKunderline*{`文 本`}
% \CJKunderdot{`文 本`} %`下划点`
% \CJKunderwave*{`文 本`}
% \CJKunderwave*{`文 本`}
% \CJKunderdblline{`文 本`}
% \CJKunderdblline*{`文 本`}
% \CJKsout{`文 本`}
% \CJKsout*{`文 本`}
% \CJKxout{`文 本`}
% \CJKxout*{`文 本`}
% \begin{CJKfilltwosides}{40mm}
%`两端分散对齐`\\
%`分散对齐`\\
%`汉字加点`\\
%\end{CJKfilltwosides}
%\end{lstlisting}
%
%\index{命令!\verb$\begin{CJKfilltwosides}$}
%
%效果如下:\\
% \CJKunderline{文 本}
% \CJKunderline*{文 本}
% \CJKunderdot{文 本}
% \CJKunderwave*{文 本}
% \CJKunderwave*{文 本}
% \CJKunderdblline{文 本}
% \CJKunderdblline*{文 本}
% \CJKsout{文 本}
% \CJKsout*{文 本}
% \CJKxout{文 本}
% \CJKxout*{文 本}\\
%\begin{CJKfilltwosides}{40mm}
%两端分散对齐\\
%分散对齐\\
%汉字加点\\
%\end{CJKfilltwosides}


\subsection{扩展符号线段 makebox}

\index{命令!\verb$\makebox$}
\index{命令!\verb$\dotfill$}
\index{命令!\verb$\hrulefill$}
\index{命令!\verb$\downbracefill$}
\index{命令!\verb$\upbracefill$}
\index{命令!\verb$\leftarrowfill$}
\index{命令!\verb$\rightarrowfill$}
\index{命令!\verb$\fbox$}
\index{命令!\verb$\shortstack$}
\index{命令!\verb$\noindent$}

\begin{shaded}
  \begin{Verbatim}
  \noindent
    \makebox[6cm]{\dotfill}\\
    \makebox[6cm]{\hrulefill}\\
    \makebox[6cm]{\downbracefill}\\
    \makebox[6cm]{\upbracefill}\\
    \makebox[6cm]{\leftarrowfill}\\
    \fbox{\shortstack{左边\\文本}}
    \rightarrowfill
    \fbox{\shortstack{右边\\文本}}
  \end{Verbatim}
\end{shaded}

效果如下:

  \noindent
    \makebox[6cm]{\dotfill}\\
    \makebox[6cm]{\hrulefill}\\
    \makebox[6cm]{\downbracefill}\\
    \makebox[6cm]{\upbracefill}\\
    \makebox[6cm]{\leftarrowfill}\\
    \fbox{\shortstack{左边\\文本}}
    \rightarrowfill
    \fbox{\shortstack{右边\\文本}}


\subsection{线段盒子}

\index{命令!\verb$\hrule$}
\index{命令!\verb$\framebox$}

主要参数有一条粗线的长度,宽度及与当前行基线的位置\\

\begin{shaded}
  \begin{Verbatim}
  \rule[垂直位移]{宽度}{高度}
    粗线\rule{6mm}{3mm}水平\\
    粗线\rule[1mm]{6mm}{3mm}水平\\
    \parbox{30mm}{Did you know \par \rule[4mm]{37mm}{1.6pt}}\\
    \framebox{\rule{9cm}{1pt}\rule{1pt}{5mm}}\\
    \framebox{\rule{9cm}{0pt}\rule{0pt}{5mm}}
  \end{Verbatim}
\end{shaded}

\noindent
    粗线\rule{6mm}{3mm}水平\\
    粗线\rule[1mm]{6mm}{3mm}水平\\
    \parbox{30mm}{Did you know?\par \rule[4mm]{37mm}{1.6pt}}\\
    \framebox{\rule{9cm}{1pt}\rule{1pt}{5mm}}\\
    \framebox{\rule{9cm}{0pt}\rule{0pt}{5mm}} %起支撑作用
